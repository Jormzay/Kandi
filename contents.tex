\section{Introduction}

\subsection{Background}

%TODO: Theoretical background, build case for the thesis based on earlier research and facts.
Financial prediction and forecasting has always been at the core of the free market model. Traditionally, predictions and pattern recognitions have been made manually by humans, who then have used this to their advantage in, for example, the stock market. New methods for gaining an edge over competitors are always being developed, and all this development nowadays take place in the realm of computer science.

Software that utilizes sophisticated algorithms, and even basic machine learning, are being used worldwide in all kinds of markets, and these methods are constantly being developed[\citenum{AlgoritmicTradeReview}]. But there has not yet been any larger adaptation of deep machine learning by any major financial player. These deep neural networks are not widely used at all in time series prediction.

If there is a niche for these deep machine learning methods, what is it, and can it compete with other contemporary methods?

%http://www.ft.com.libproxy.aalto.fi/intl/cms/s/2/019b3702-92a2-11e4-a1fd-00144feabdc0.html#axzz3St427BVm

%Financial times article regarding investments into machine learning in 2014


\subsection{Problem description}

%TODO: Describe and explain the research angle (problem) carefully

The different possible uses for deep learning in analysing and predicting financial time series have not been as well documented as other methods. This thesis is looking at the results of studies done regarding the usage and performance of a specific set of deep learning method applied in financial data series. The goal is to be able to draw some overreaching conclusions as the possibilities and efficiency of these deep learning methods compared to other contemporary and more broadly used ones. Results are approached mainly from a business perspective, meaning that the competitive power and monetary value creation potential of a prediction method is valued more than the research and scientific value of it.


\subsection{Problem definition and purpose}
%TODO: Formulate the problem further and define the purpose of the thesis. Present research questions.

The purpose, as briefly stated in the previous sub-chapter, is to assess the business potential and competitiveness of a specific deep learning method when used for forecasting in a financial market. There are a number of questions very central to this purpose, that are reiterated in one form or another troughout this thesis.

\begin{itemize}

\item Is the particular deep NN method comparable to other contemporary methods?

\item ?

\end{itemize}

On top of this, some back of the envelope calculations are presented regarding the actual costs of designing, setting up and running a financial operation based upon the methods discussed in this thesis. This will amount to a very rudimentary business case for a hypothetical entrepreneur.


%Viability in business?

%Curiosity or opportunity?

\subsection{Subject Restrictions}
%TODO: Restrict the field of research, and motivate the restrictions.

The conclusions drawn in this thesis are based on one singular method of deep machine learning, and in only a few financial markets. The deep learning method discussed is explained to such an extent that the conclusion can be understood. That is to say, that technical details are not brought up more than necessary, and a deeper understanding of machine learning and neural networks is not to required to read, nor achieved by reading, this thesis. This is to leave room for and focus on the actual results produced by the methods, and the actual analysis of these results.

The financial markets dealt with are restricted to a select few, based on available literature and research. Concepts related to finance and economics are explained in a contextual manner, no in-depth explanation of theories. Again, the interpretation of results and the formulation of some kind of conclusion supersedes the need for extensive explaining and rigorous dissection of the science involved.

The actual raw and preprocessed financial data the machine learning methods use will not be discussed more than to name their source and, if relevant, how the data has been processed. The gathering and utilization of large data samples is a science in and of itself, a science not included in this work.


\subsection{Material and methods}

%TODO: Describe the research in sufficient detail so that reproduction and/or falsification is possible by third party. Discuss the main sources and elevate characteristics such as problems, shortcomings and usefulness.

A challenge, compared to other fields, when searching for information is undoubtedly that this particular subject does not yet boast a large body of research. Relevant articles and scientific papers are scarce, and often generalized in such ways that they seldom contribute to the purpose of this thesis. In order to get enough reliable data to speculate and draw conclusions on, some modest cross-field extrapolations have been made. This is to say, that wherever deep neural networks have been applied to datasets resembling that of financial sets, the results have been used to argue the some conclusion regarding the finance


\newpage

\section{Results}
%TODO: Contents of Results chapter
\subsection{Deep NN in equivalent datasets}
%Scarcity of information --> extrapolate results from similar datasets

%http://cs229.stanford.edu/proj2012/BussetiOsbandWong-DeepLearningForTimeSeriesModeling.pdf

%Get Citation on underlying non-linear processes and correlations. Complicated processes.
\newpage

\section{A Curiosity vs An Opportunity}
TODO: Analysis and Interpretation

\newpage

\section{Conclusion}
TODO: Conclusion

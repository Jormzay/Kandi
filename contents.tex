\section{Introduction}

\subsection{Background and problem description}

%TODO: Theoretical background, build case for the thesis based on earlier research and facts. Describe and explain the problem.

Analysis of financial data is a broad subject and very fundamental in the field of economics. Contemporary types and methods for such analysis are numerous, but a large subcategory is focused the detection of trends in financial time series, formerly known as financial forecasting.

Financial forecasting is a major part of the modern free market model. Software that utilizes sophisticated algorithms, and even basic machine learning, are being used worldwide in all kinds of markets, and these methods are continuously being perfected and developed. [\citenum{AlgoritmicTradeReview}]. Most of these algorithms and computer methods deal with short term trends and take advantage of a computers' fast reaction time compared to that of human beings. Having a computer make predictions in the short term, however, is very different than in the long term, as macroeconomic, political and environmental factors don't generally have to be taken into account. These are factors that have a significant impact in the long term, require human interpretation, and further obscure any underlying trends that an automated computer model might pick up.\citenum{FForecastingIntro}

The question is then, are there viable options for automated long term financial forecasting, and are they at all comparable in terms of efficiency and success to contemporary forecasting, automated or not.

Forecasting of larger datasets, especially time series, is often accomplished using machine learning and, more specifically, so called artificial neural networks. A subcategory of machine learning is the field of deep learning and deep neural networks, which is for more complex pattern recognition.

This begs the question, could a deep neural network discover subtle long term trends in a financial data time series, buried in the noise of the global economy, that general neural networks would not be able to detect?


\subsection{Definitions and purpose of thesis}
%TODO: Formulate the problem further and define the purpose of the thesis. Present research questions.


A deep neural network is a method for machine learning, that in short is a neural network with several layers that define different levels of abstraction. The output of a neural network is used directly as input in the next network, marking two different abstract levels. A deep NN is generally better at recognising more complex and deep patterns, at the cost of efficiency and computing power.

Financial data in this context means any temporal data of sufficient quantity related to economics. An example would be a stock market index over time, or any market index in general. Furthermore, as financial data consist mainly of time series, any temporal data set with similar features could be used for testing a forecasting method. More on that in the Material and methods chapter.

The purpose, as hinted at in the previous sub-chapter, is to assess the viability of using deep neural networks when used for forecasting in a financial data set. The goal is to be able to draw some overreaching conclusions as to the possibilities and efficiency of these deep learning methods, and see which method in particular is well suited for this data.

\subsection{Focus and subject restrictions}
%TODO: Restrict the field of research, and motivate the restrictions.

The subject of this is thesis is restricted to an overview of the viability of financial forecasting using deep neural network methods. Technical intricacies and specifications of deep neural networks and deep learning in general are kept to a minimum, and the different methods covered are only analysed in the context of financial data.

The conclusions are conveyed a summarized manner, with the goal of giving a cohesive statement instead of numerous small points. This will further restrict the depth the analysis.

No knowledge of either machine learning nor finance is supposed of the reader.




\subsection{Research material and methods}

%TODO: Describe the research in sufficient detail so that reproduction and/or falsification is possible by third party. Discuss the main sources and elevate characteristics such as problems, shortcomings and usefulness.

This thesis is mainly a materials study, with analysis and conclusions deducted from existing work.

The type of material used is ranging from published papers and studies to university student reports.



\newpage

\section{Existing research and results}


\subsection{Current state of research}

Here, first the characteristics of financial and other time series forecasting will be discussed, and secondly relevant papers and research material will be presented.

A challenge with the field of deep learning, compared to other fields, when searching for information is undoubtedly that this particular subject does not yet boast a large body of research. Relevant articles and scientific papers are scarce, and often generalized in such ways that they rarely contribute to the specific purpose of this thesis. 

As a result of this, the papers discussed are chosen simply on the one merit that they have at something to bring to the discussion, however lacking they otherwise may be.

Furthermore, since finding papers related to exactly financial data and deep neural networks is difficult, some extrapolation can be made from similar fields with similar data structures.


\subsection{Characteristics of deep neural networks}

The strength in deep machine learning is the ability to identify and account for fundamental non linear patterns in large data sets.

Financial data sets are inherently noisy, because they can not contain the full spectrum of data needed to make an 100\% accurate prediction. The missing data, such as market rumours and complex political and social factors are considered 'noise', because they can't be modelled. Furthermore, the data is considered dynamic, and even chaotic, in the sense that the distribution of a financial series changes over time, rendering assumptions based on old sets irrelevant. [\citenum{FForecastingIntro}]

These are all characteristics that a deep neural network will need to cope with, in order to compete with other models. So when analysing model performance, the  specific type of data is not really relevant, as long as it matches these criteria.

\subsection{Deep neural networks forecast in electricity demand}

%More on the study, who, what, where etc.
%Some formulas or Graphs perhaps? More technical details, but not too much!

In a project report done at Stanford university in 2012[\citenum{StanfordNNReport}], a deep NN method was employed to forecast energy demand in the US. This kind of data, of which there are huge, easily accessible data sets, is particularly well suited for machine learning purposes. Furthermore, energy load data exhibit deep non-linear patterns, much like financial data does.

The study itself concluded in favour of deep learning techniques. With a Deep Recurrent NN, they obtained as high correlation rates as 99.6\%. In comparison, top industrial forecasters advertise mean percentage errors of between 0.84\% and 1.56\%. This puts the Deep NN on par with the conventional methods of the private sector.

The Deep recurrent NN responsible for the results, outperformed the other deep NN methods (ie. Deep Feedforward NN) used in the study by a wide margin. The authors point out that the other methods fail to incorporate some of the temporal structures found in time series data. These are data structures related to time intervals and points, essential to forecasting time series. This is of course relevant for other time series as well, such as financial market data.

In their conclusion, the authors contribute parts of their results to the nature of the data used, and the availability of it. Furthermore they state that by refining of their methods, and more specifically, by incorporating more advanced techniques in the process, the results could still improve. They specifically name the restricted Boltzmann machines as a more modern method they believe could out-perform their own top contender, a deep recurrent NN.


%Research Restricted Boltzmann machines
%Get Citation on underlying non-linear processes and correlations. Complicated processes.


\newpage

\section{Analysis}

\subsection{Analytical approach}

In this section the papers previously presented are dissected and their potentially relevant parts pointed out.
Each piece of research is analysed on its own, and the main points not put together until the conclusion section of the thesis.
 


\subsection{Comparing different Deep NNs}

\subsection{Deep NNs with contemporary linear methods}


\newpage

\section{Conclusion}

\subsection{Summary of analysis}

\subsection{Future development and possible outlooks}


\section{Introduction}

\subsection{Background}

%TODO: Theoretical background, build case for the thesis based on earlier research and facts.
Financial prediction and forecasting has always been at the core of the free market model. Traditionally, predictions and pattern recognitions have been made manually by humans, who then have used this to their advantage in, for example, the stock market. New methods for gaining an edge over competitors are always being developed, and all this development nowadays take place in the realm of computer science.

Software that utilizes sophisticated algorithms, and even basic machine learning, are being used worldwide in all kinds of markets, and these methods are constantly being developed[\citenum{AlgoritmicTradeReview}]. But there has not yet been any larger adaptation of deep machine learning by any major financial player. These deep neural networks are not widely used at all in time series prediction.

If there is a niche for these deep machine learning methods, what is it, and can it compete with other contemporary methods?


\subsection{Problem description}

%TODO: Describe and explain the research angle (problem) carefully

The different possible uses for deep learning in analysing and predicting financial time series have not been as well documented as other methods. This thesis is looking at the results of studies done regarding the usage and performance of a specific set of deep learning method applied in financial data series. The goal is to be able to draw some overreaching conclusions as the possibilities and efficiency of these deep learning methods compared to other contemporary and more broadly used ones. Results are approached mainly from a business perspective, meaning that the competitive power and monetary value creation potential of a prediction method is valued more than the research or scientific value of it.


\subsection{Problem definition and purpose}
%TODO: Formulate the problem further and define the purpose of the thesis. Present research questions.

The purpose, as briefly stated in the previous sub-chapter, is to assess the business potential and competitiveness of a specific deep learning method when used for forecasting in a financial market. There are a number of questions very central to this purpose, that are reiterated in one form or another troughout this thesis.

\begin{itemize}

\item Is the method viable for business in a specific financial context?

\item Will the method be regarded as an opportunity or as a curiosity?

\end{itemize}

On top of this, some back of the envelope calculations are presented regarding the actual costs of designing, setting up and running a financial operation based upon the methods discussed in this thesis. This will amount to a very rudimentary business case for a hypothetical entrepreneur.


%Viability in business?

%Curiosity or opportunity?

\subsection{Restrictions}
%TODO: Restrict the field of research, and motivate the restrictions.

The conclusions drawn in this thesis are based on one case of deep machine learning, in only a few financial markets.


\subsection{Material and methods}
%TODO: Describe the research in sufficient detail so that reproduction and/or falsification is possible by third party. Discuss the main sources and elevate characteristics such as problems, shortcomings and usefulness.

\newpage

\section{Title for Results chapter}
%TODO: Contents of Results chapter
\subsection{Titles for any number of subchapters}

\newpage

\section{Analysis and Interpretation}
TODO: Analysis and Interpretation

\newpage

\section{Conclusion}
TODO: Conclusion

\section{Introduction}

\subsection{Background}

%TODO: Theoretical background, build case for the thesis based on earlier research and facts.


Financial forecasting is a corned stone of the modern free market model. Software that utilizes sophisticated algorithms, and even basic machine learning, are being used worldwide in all kinds of markets, and these methods are continuously being perfected and developed. [\citenum{AlgoritmicTradeReview}]. But there has not yet been any larger adaptation of deep machine learning in financial forecasting. These deep neural networks are in fact not widely used at all in time series prediction in general.

A deep neural network is a method for machine learning, that in short is a neural network with several layers that define different levels of abstraction. The output of a neural network is used directly as input in the next network, marking two different abstract levels. A deep NN is generally better at recognising more complex and deep pattern, at the cost of efficiency and computing power.

What this thesis is aiming for, is to assess whether deep NN methods can compete with other contemporary techniques of financial forecasting.


\subsection{Problem description}

%TODO: Describe and explain the research angle (problem) carefully

The different possible uses for deep learning in analysing and predicting financial time series have not been as well documented as other methods. As a result of this, no apparent consensus can be found to indicate one way or the other.

This thesis is looking at the results of studies done regarding the usage and performance of deep learning methods applied in financial data series. The goal is to be able to draw some overreaching conclusions as to the possibilities and efficiency of these deep learning methods compared to other contemporary and more broadly used ones. Results are analysed such that the current, somewhat early state of deep NNs is taken into account.


\subsection{Problem definition and purpose}
%TODO: Formulate the problem further and define the purpose of the thesis. Present research questions.

The purpose, as briefly stated in the previous sub-chapter, is to assess the business potential and competitiveness of using deep neural networks (NN) when used for forecasting in a financial market. There are a number of questions very central to this purpose, that are reiterated in one form or another troughout this thesis.

\begin{itemize}

\item Is any particular deep learning method comparable to other contemporary methods?

\item What aspect of finance forecasting is most suited for deep NNs?

\item What kind of developments is realistically to be expected within the field in the near future, and how should it be taken into account when formulating conclusions.

\end{itemize}




\subsection{Subject Restrictions}
%TODO: Restrict the field of research, and motivate the restrictions.

The subject of this is thesis is restricted to a literature study regarding the viability of financial forecasting using neural networks. No knowledge of either machine learning nor finance is supposed of the reader, so the technical details are kept to a minimum.
The different types of NNs and deep NNs will not be discussed any more than is contextually necessary.
Analysis of results gathered is focused on the end-result, and not for example on the efficiency or energy consumption of methods and techniques.




\subsection{Material and methods}

%TODO: Describe the research in sufficient detail so that reproduction and/or falsification is possible by third party. Discuss the main sources and elevate characteristics such as problems, shortcomings and usefulness.

A challenge, compared to other fields, when searching for information is undoubtedly that this particular subject does not yet boast a large body of research. Relevant articles and scientific papers are scarce, and often generalized in such ways that they seldom contribute to the specific purpose of this thesis. In order to get enough reliable data to speculate and draw conclusions on, some modest cross-field extrapolations are made.

The type of material used is ranging from published papers and studies to university student reports.



\newpage

\section{Results}
%TODO: Contents of Results chapter
\subsection{Deep NN in equivalent datasets}
%Scarcity of information --> extrapolate results from similar datasets

Here the basic characteristics of financial and other time series forecasting in general will be discussed.
The strength in deep machine learning is the ability to identify and account for fundamental non linear patterns in large data sets.

Financial data sets are inherently noisy, because they can not contain the full spectrum of data needed to make an 100\% accurate prediction. The missing data, such as market rumours and complex political and social factors are considered 'noise', because they can't be modelled. Furthermore, the data is considered dynamic, and even chaotic, in the sense that the distribution of a financial series changes over time, rendering assumptions based on old sets irrelevant. \citenum{FForecastingIntro}

These are all characteristics that a deep neural network will need to cope with, in order to compete with other models. So when analysing model performance, the  specific type of data is not really relevant, as long as it matches these criteria. This is especially useful in this case, when relevant studies and papers are scarce.

In the next section we will extrapolate the results from using deep NN to predict energy demand, to draw conclusion that are valid in financial forecasting as well.

\subsection{Deep NN in times series forecasting}

In a project report done at Stanford university in 2012[\citenum{StanfordNNReport}], a deep NN method was employed to forecast energy demand in the US. This kind of data, of which there are huge easily accesible data sets, is particularly well suited for machine learning purposes. Furthermore, energy load data exhibit deep non-linear patterns, much like financial data does, and it is considered possible to draw on the conclusions reached in this paper in assessing deep NN in the case of financial data.

The study itself concluded in favour of deep learning techniques. With a Deep Recurrent NN, they obtained as high correlation rates as 99.6\%. In comparison, top industrial forecasters advertise mean percentage errors of between 0.84\% and 1.56\%. This puts the Deep NN on par with the conventional methods of the private sector.

The Deep recurrent NN responsible for the results, outperformed the other deep NN methods (ie. Deep Feedforward NN) used in the study by a wide margin. The authors point out that the other methods fail to incorporate some of the temporal structures found in time series data. These are data structures related to time intervals and points, essential to forecasting time series. This is of course relevant for other time series as well, such as financial market data.

In their conclusion, the authors contribute parts of their results to the nature of the data used, and the availability of it. Furthermore they state that by refining of their methods, and more specifically, by incorporating more advanced techniques in the process, the results could still improve. They specifically name the restricted Boltzmann machines as a more modern method they believe could out-perform their own top contender, a deep recurrent NN.


%Research Restricted Boltzmann machines
%Get Citation on underlying non-linear processes and correlations. Complicated processes.


\newpage

\section{Analysis}
TODO: Analysis and Interpretation

\newpage

\section{Conclusion}
TODO: Conclusion

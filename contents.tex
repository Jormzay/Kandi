\section{Introduction}

\subsection{Background}

%TODO: Theoretical background, build case for the thesis based on earlier research and facts.
Financial prediction and forecasting has always been at the core of the free market model. Traditionally, predictions and pattern recognitions have been made manually by humans, who then have used this to their advantage in, for example, the stock market. New methods for gaining an edge over competitors are always being developed, and all this development nowadays take place in the realm of computer science.

Software that utilizes sophisticated algorithms, and even basic machine learning, are being used worldwide in all kinds of markets, and these methods are constantly being developed[\citenum{AlgoritmicTradeReview}]. But there has not yet been any larger adaptation of deep machine learning by any major financial player. These deep neural networks are not widely used at all in time series prediction.

If there is a niche for these deep machine learning methods, what is it, and can it compete with other contemporary methods?


\subsection{Problem description}

%TODO: Describe and explain the research angle (problem) carefully

The different possible uses for deep learning in analysing and predicting financial time series have not been as well documented as other methods. This thesis will look at the results of different 
Deep Neural networks for time series - for finances.
Large dataset, non-linear patterns abundant.
Vs contemporary methods?


\subsection{Problem definition and purpose}
%TODO: Formulate the problem further and define the purpose of the thesis. Present research questions.

Deep learning vs contemporary methods?

Viability in business?

Curiosity or opportunity?

\subsection{Restrictions}
%TODO: Restrict the field of research, and motivate the restrictions.


\subsection{Material and methods}
%TODO: Describe the research in sufficient detail so that reproduction and/or falsification is possible by third party. Discuss the main sources and elevate characteristics such as problems, shortcomings and usefulness.

\newpage

\section{Title for Results chapter}
%TODO: Contents of Results chapter
\subsection{Titles for any number of subchapters}

\newpage

\section{Analysis and Interpretation}
TODO: Analysis and Interpretation

\newpage

\section{Conclusion}
TODO: Conclusion

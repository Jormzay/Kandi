% ---------------------------------------------------------------------
% -------------- PREAMBLE ---------------------------------------------
% ---------------------------------------------------------------------
\documentclass[12pt,a4paper,finnish,oneside]{article}
%\documentclass[12pt,a4paper,finnish,twoside]{article}
%\documentclass[12pt,a4paper,finnish,oneside,draft]{article} % luonnos, nopeampi

% Valitse 'input encoding':
%\usepackage[latin1]{inputenc} % merkistökoodaus, jos ISO-LATIN-1:tä.
\usepackage[utf8]{inputenc}   % merkistökoodaus, jos käytetään UTF8:a
% Valitse 'output/font encoding':
%\usepackage[T1]{fontenc}      % korjaa ääkkösten tavutusta, bittikarttana
\usepackage{ae,aecompl}       % ed. lis. vektorigrafiikkana bittikartan sijasta
% Kieli- ja tavutuspaketit:
\usepackage[english,swedish,finnish]{babel}
% Kurssin omat asetukset aaltosci_t.sty:
%\usepackage{aaltosci_t}
% Jos kirjoitat muulla kuin suomen kielellä valitse:
%\usepackage[finnish]{aaltosci_t}           
%\usepackage[swedish]{aaltosci_t}           
\usepackage[english]{aaltosci_t}           
% Muita paketteja:
\usepackage{alltt}
\usepackage{amsmath}   % matematiikkaa
\usepackage{calc}      % käytetään laskurien (counter) yhteydessä (tiedot.tex)
\usepackage{eurosym}   % eurosymboli: \euro{}
\usepackage{url}       % \url{...}
\usepackage{listings}  % koodilistausten lisääminen
\usepackage{algorithm} % algoritmien lisääminen kelluvina
\usepackage{algorithmic} % algoritmilistaus
\usepackage{hyphenat}  % tavutuksen viilaamiseen liittyvä (hyphenpenalty,...)
\usepackage{supertabular,array}  % useampisivuinen taulukko

% Koko dokumentin kattavia asetuksia:

% Tavutettavia sanoja:
%\hyphenation{vää-rin me-ne-vi-en eri-kois-ten sa-no-jen tavu-raja-ehdo-tuk-set}
% Huomaa, että ylläoleva etsii tarkalleen kyseisiä merkkijonoja, eikä
% ymmärrä taivutuksia. Paikallisesti tekstin seassa voi myös ta\-vut\-taa.
%���asd������l���
% Rangaistaan tavutusta (ei toimi?! Onko hyphenat-paketti asennettu?)
\hyphenpenalty=10000   % rangaistaan tavutuksesta, 10000=ääretön
\tolerance=1000        % siedetään välejä riveillä
% titlesec-paketti auttaa, jos tämän mukana menee sekaisin

% Tekstiviitteiden ulkoasu.
% Pakettiin natbib.sty/aaltosci.bst liittyen katso esim. 
% http://merkel.zoneo.net/Latex/natbib.php
% jossa selitykset citep, citet, bibpunct, jne.
% Valitse alla olevista tai muokkaa:
%\bibpunct{(}{)}{;}{a}{,}{,}    % a = tekijä-vuosi (author-year)
\bibpunct{[}{]}{;}{n}{,}{,}    % n = numero [1],[2] (numerical style)

% Rivivälin muuttaminen:
\linespread{1.24}\selectfont               % riviväli 1.5
%\linespread{1.24}\selectfont               % riviväli 1, kun kommentoit pois

% ---------------------------------------------------------------------
% -------------- DOCUMENT ---------------------------------------------
% ---------------------------------------------------------------------

\begin{document}

% -------------- Tähän dokumenttiin liittyviä valintoja  --------------

%\raggedright         % Tasattu vain vasemmalta, ei tavutusta
\input{makroja}       % Haetaan joitakin makroja

% Kieli:
% Kielesi, jolla kandidaatintyön kirjoitat: finnish, swedish, english.
% Tästä tulee mm. tietyt otsikkonimet ja kuva- ja taulukkoteksteihin 
% (Kuva, Figur, Figure), (Taulukko, Tabell, Table) sekä oikea tavutus.
%\selectlanguage{finnish}
%\selectlanguage{swedish}
\selectlanguage{english}

% Sivunumeroinnin kanssa pieniä ristiriitaisuuksia.
% Toimitaan pääosin lähteen "Kirjoitusopas" luvun 5.2.2 mukaisesti.
% Sivut numeroidaan juoksevasti arabialaisin siten että 
% ensimmäiseltä nimiölehdeltä puuttuu numerointi.
%\pagestyle{plain}
%\pagenumbering{arabic}
% Muita tapoja: kandiohjeet: ei numerointia lainkaan ennen tekstiosaa
%\pagestyle{empty}
% Muita tapoja: kandiohjeet: roomalainen numerointi alussa ennen tekstiosaa
\pagestyle{plain}
\pagenumbering{roman}        % i,ii,iii, samalla alustaa laskurin ykköseksi

% ---------------------------------------------------------------------
% -------------- Luettelosivut alkavat --------------------------------
% ---------------------------------------------------------------------

% -------------- Nimiölehti ja sen tiedot -----------------------------
%
% Nimiölehti ja tiivistelmä kirjoitetaan seminaarin mukaan joko
% suomeksi tai ruotsiksi (ellei erityisesti kielenä ole englanti). 
% Tiivistelmän voi suomen/ruotsin lisäksi kirjoittaa halutessaan
% myös englanniksi. Eli tiivistelmiä tulee yksi tai kaksi kpl.
%
% "\MUUTTUJA"-kohdat luetaan aaltosci_t.sty:ä varten.

\author{Jonathan Rehn}

% Otsikko nimiölehdelle. Yleensä sama kuin seuraavana oleva \TITLE, 
% mutta jos nimiölehdellä tarvetta "kaksiosaiselle" kaksiriviselle
\title{Deep Neural Networks for Financial Data}
% 2-osainen otsikko:
%\title{\LaTeX{}-pohja kandidaatintyölle \\[5mm] Pitkiä rivejä kokeilun vuoksi.}

% Otsikko tiivistelmään. Jos lisäksi engl. tiivistelmä, niin viimeisin:
%\TITLE{Deep Neural Networks for Financial Data}
%\TITLE{\LaTeX{} för kandidatseminariet med jättelång rubrik som fortsätter och
% fortsätter ännu}
\ENTITLE{Deep Neural networks for financial data}
% 2-osainen otsikko korvataan täällä esim. pisteellä:
%\TITLE{\LaTeX{}-pohja kandidaatintyölle. Pitkiä rivejä kokeilun vuoksi.}

% Ohjaajan laitos suomi/ruotsi ja tarvittaessa eng (tiivistelmän kieli/kielet)
% suomi:
%\DEPT{Tietotekniikan laitos}               % T
%\DEPT{Tietojenkäsittelytieteen laitos}     % TKT
%\DEPT{Mediatekniikan laitos}               % ME
% ruotsi:
%\DEPT{Institutionen för datateknik}        % T
%\DEPT{Institutionen för datavetenskap}     % TKT
%\DEPT{Institutionen för mediateknik}       % ME
% englanti:
\ENDEPT{Department of Computer Science Engineering}     % T
%\ENDEPT{Department of Information and Computer Science} % TKT
%\ENDEPT{Department of Media Technology}                 % ME

% Vuosi ja päivämäärä, jolloin työ on jätetty tarkistettavaksi.
\YEAR{2015}
%\DATE{27. xxxxxkuuta 20xx}
\ENDATE{February 27, 2015}


%\INSTRUCTOR{Ohjaajantitteli Sinun Ohjaajasi, ToinenTitt Matti Meikäläinen}
% DI       // på svenska DI diplomingenjör
% TkL      // TkL teknologie licentiat
% TkT      // TkD teknologie doctor
% Dosentti Dos. // Doc. Docent
% Professori Prof. // Prof. Professor
% 
% Jos tiivistelmä englanniksi, niin:
\ENSUPERVISOR{Prof. Olof Forsen}
\ENINSTRUCTOR{Pyry Takala}
% M.Sc. (Tech)  // M.Sc. (Eng)
% Lic.Sc. (Tech)
% D.Sc. (Tech)   // FT filosofian tohtori, PhD Doctor of Philosophy
% Docent
% Professor

% Kirjoita tänne HOPS:ssa vahvistettu pääaineesi.
% Pääainekoodit TIK-opinto-oppaasta.

\PAAAINE{Computer Science}
\CODE{SCI3027}


%\PAAAINE{Ohjelmistotuotanto ja -liiketoiminta}
%\CODE{T3003}
%
%\PAAAINE{Tietoliikenneohjelmistot}
%\CODE{T3005}
%
%\PAAAINE{WWW-teknologiat} % vuodesta 2010
%\CODE{IL3012}
%
%\PAAAINE{Mediatekniikka} % vuoteen 2010, kts. seur.
%\CODE{T3004}
%
%\PAAAINE{Mediatekniikka} % vuodesta 2010, kts. edell.
%\CODE{IL3011}
%
%\PAAAINE{Tietojenkäsittelytiede} % vuodesta 2010
%\CODE{IL3010}
%
%\PAAAINE{Informaatiotekniikka} % vuoteen 2010
%\CODE{T3006}
%
%\PAAAINE{Tietojenkäsittelyteoria} % vuoteen 2010
%\CODE{T3002}
%
%\PAAAINE{Ohjelmistotekniikka}
%\CODE{T3001}

% Avainsanat tiivistelmään. Tarvittaessa myös englanniksi:

%\KEYWORDS{avain, sanoja, niitäkin, tähän, vielä, useampi, vaikkei, %
%niitä, niin, montaa, oikeasti, tarvitse}
\ENKEYWORDS{Nerual networks, Financial prediction, Deep machine learning}

% Tiivistelmään tulee opinnäytteen sivumäärä.
% Kirjoita lopulliset sivumäärät käsin tai kokeile koodia. 
%
% Ohje 29.8.2011 kirjaston henkilökunnalta:
%   - yhteissivumäärä nimiölehdeltä ihan loppuun
%   - "kaikkien yksinkertaisin ja yksiselitteisin tapa"

\PAGES{X}
%\PAGES{23}  % kaikki sivut laskettuna nimiölehdestä lähdeluettelon tai 
             % mahdollisten liitteiden loppuun. Tässä 23 sivua

%\thispagestyle{empty}  % nimiölehdellä ei ole sivunumerointia; tyylin mukaan ei tehdäkään?!

\maketitle             % tehdään nimiölehti

% -------------- Tiivistelmä / abstract -------------------------------
% Lisää abstrakti kandikielellä (ja halutessasi lisäksi englanniksi).

% Edelleen sivunumerointiin. Eräs ohje käskee aloittaa sivunumeroiden
% laskemisen nimiösivulta kuitenkin niin, että sille ei numeroa merkitä
% (Kauranen, luku 5.2.2). Näin ollen ensimmäisen tiivistelmän sivunumero
% on 2. \maketitle komento jotenkin kadottaa sivunumeronsa.
\setcounter{page}{2}    % sivunumeroksi tulee 2

% Tiivistelmät tehdään viimeiseksi. 
%
% Tiivistelmä kirjoitetaan käytetyllä kielellä (JOKO suomi TAI ruotsi)
% ja HALUTESSASI myös samansisältöisenä englanniksi.
%
% Avainsanojen lista pitää merkitä main.tex-tiedoston kohtaan \KEYWORDS.


%Tiivistelmän tyypillinen rakenne: 
%(1) aihe, tavoite ja rajaus 
%(heti alkuun, selkeästi ja napakasti, ei johdattelua);
%(2) aineisto ja menetelmät (erittäin lyhyesti);
%(3) tulokset (tälle enemmän painoarvoa); 
%(4) johtopäätökset (tälle enemmän painoarvoa).
%
%Tiivistelmätekstiä tähän (\languagename). Huomaa, että tiivistelmä tehdään %vasta kun koko työ on muuten kirjoitettu.


%\begin{svabstract}
%  Ett abstrakt hit 
%%(\languagename)
%\end{svabstract}

\begin{enabstract}

TODO: Abstract

% Here goes the abstract 
%%(\languagename)
\end{enabstract}

\newpage                       % pakota sivunvaihto

% -------------- Sisällysluettelo / TOC -------------------------------

\tableofcontents

\label{pages:prelude}
\clearpage                     % kappale loppuu, loput kelluvat tänne, sivunv.
%\newpage

% -------------- Symboli- ja lyhenneluettelo -------------------------
% Lyhenteet, termit ja symbolit.
% Suositus: Käytä vasta kun paljon symboleja tai lyhenteitä.
%
%\input{luku_lyhenteet} 
%\clearpage                     % luku loppuu, loput kelluvat tänne
\newpage

% -------------- Kuvat ja taulukot ------------------------------------
% Kirjoissa (väitöskirja) on usein tässä kuvien ja taulukoiden listaus.
% Suositus: Ei kandityöhön.

% -------------- Alkusanat --------------------------------------------
% Suositus: ÄLÄ käytä kandidaatintyössä. Jos käytät, niin omalle 
% sivulleen käyttäen tarvittaessa \newpage
%
%\input{luku_alkusanat}
%\clearpage                     % luku loppuu, loput kelluvat tänne
%\newpage                       % pakota sivunvaihto
%
%SH: Alkusanoissa voi kiittää tahoja, jotka ovat merkittävästi edistäneet
% työn valmistumista. Tällaisia voivat olla esimerkiksi yritys, jonka
% tietokantoja, kontakteja tai välineistöä olet saanut käyttöösi,
% haastatellut henkilöt, ohjaajasi tai muut opettajat ja myös
% henkilökohtaiset kontaktisi, joiden tuki on ollut korvaamatonta työn
% kirjoitusvaiheessa. Alkusanat jätetään tyypillisesti pois
% kandidaatintyöstä, joka on laajuudeltaan vielä niin suppea, ettei
% kiiteltäviä tahoja luontevasti ole.

% ---------------------------------------------------------------------
% -------------- Tekstiosa alkaa --------------------------------------
% ---------------------------------------------------------------------

% Muutetaan tarvittaessa ala- ja ylätunnisteet
%\pagestyle{headings}          % headeriin lisätietoja
%\pagestyle{fancyheadings}     % headeriin lisätietoja
%\pagestyle{plain}             % ei header, footer: sivunumero

% Sivunumerointi, jos käytetty 'roman' aiemmin
\pagenumbering{arabic}        % 1,2,3, samalla alustaa laskurin ykköseksi
% \thispagestyle{empty}         % pyydetty ensimmäinen tekstisivu tyhjäksi

% input-komento upottaa tiedoston 
%\input{luku_sisalto}
\section{Introduction}

\subsection{Background and problem description}

%TODO: Theoretical background, build case for the thesis based on earlier research and facts. Describe and explain the problem.

Analysis of financial data is a broad subject and very fundamental in the field of economics. Contemporary types and methods for such analysis are numerous, but a large subcategory is focused the detection of trends in financial time series, formerly known as financial forecasting.

Financial forecasting is a major part of the modern free market model. Software that utilizes sophisticated algorithms, and even basic machine learning, are being used worldwide in all kinds of markets, and these methods are continuously being perfected and developed. [\citenum{AlgoritmicTradeReview}]. Most of these algorithms and computer methods deal with short term trends and take advantage of a computers' fast reaction time compared to that of human beings. Having a computer make predictions in the short term, however, is very different than in the long term, as macroeconomic, political and environmental factors don't generally have to be taken into account. These are factors that have a significant impact in the long term, require human interpretation, and further obscure any underlying trends that an automated computer model might pick up.\citenum{FForecastingIntro}

The question is then, are there viable options for automated long term financial forecasting, and are they at all comparable in terms of efficiency and success to contemporary forecasting, automated or not.

Forecasting of larger datasets, especially time series, is often accomplished using machine learning and, more specifically, so called artificial neural networks. A subcategory of machine learning is the field of deep learning and deep neural networks, which is for more complex pattern recognition.

This begs the question, could a deep neural network discover subtle long term trends in a financial data time series, buried in the noise of the global economy, that general neural networks would not be able to detect?


\subsection{Definitions and purpose of thesis}
%TODO: Formulate the problem further and define the purpose of the thesis. Present research questions.


A deep neural network is a method for machine learning, that in short is a neural network with several layers that define different levels of abstraction. The output of a neural network is used directly as input in the next network, marking two different abstract levels. A deep NN is generally better at recognising more complex and deep patterns, at the cost of efficiency and computing power.

Financial data in this context means any temporal data of sufficient quantity related to economics. An example would be a stock market index over time, or any market index in general. Furthermore, as financial data consist mainly of time series, any temporal data set with similar features could be used for testing a forecasting method. More on that in the Material and methods chapter.

The purpose, as hinted at in the previous sub-chapter, is to assess the viability of using deep neural networks when used for forecasting in a financial data set. The goal is to be able to draw some overreaching conclusions as to the possibilities and efficiency of these deep learning methods, and see which method in particular is well suited for this data.

\subsection{Focus and subject restrictions}
%TODO: Restrict the field of research, and motivate the restrictions.

The subject of this is thesis is restricted to an overview of the viability of financial forecasting using deep neural network methods. Technical intricacies and specifications of deep neural networks and deep learning in general are kept to a minimum, and the different methods covered are only analysed in the context of financial data.

The conclusions are conveyed a summarized manner, with the goal of giving a cohesive statement instead of numerous small points. This will further restrict the depth the analysis.

No knowledge of either machine learning nor finance is supposed of the reader.




\subsection{Research material and methods}

%TODO: Describe the research in sufficient detail so that reproduction and/or falsification is possible by third party. Discuss the main sources and elevate characteristics such as problems, shortcomings and usefulness.

This thesis is mainly a materials study, with analysis and conclusions deducted from existing work.

The type of material used is ranging from published papers and studies to university student reports.



\newpage

\section{Existing research and results}


\subsection{Current state of research}

Here, first the characteristics of financial and other time series forecasting will be discussed, and secondly relevant papers and research material will be presented.

A challenge with the field of deep learning, compared to other fields, when searching for information is undoubtedly that this particular subject does not yet boast a large body of research. Relevant articles and scientific papers are scarce, and often generalized in such ways that they rarely contribute to the specific purpose of this thesis. 

As a result of this, the papers discussed are chosen simply on the one merit that they have at something to bring to the discussion, however lacking they otherwise may be.

Furthermore, since finding papers related to exactly financial data and deep neural networks is difficult, some extrapolation can be made from similar fields with similar data structures.


\subsection{Characteristics of deep neural networks}

The strength in deep machine learning is the ability to identify and account for fundamental non linear patterns in large data sets.

Financial data sets are inherently noisy, because they can not contain the full spectrum of data needed to make an 100\% accurate prediction. The missing data, such as market rumours and complex political and social factors are considered 'noise', because they can't be modelled. Furthermore, the data is considered dynamic, and even chaotic, in the sense that the distribution of a financial series changes over time, rendering assumptions based on old sets irrelevant. [\citenum{FForecastingIntro}]

These are all characteristics that a deep neural network will need to cope with, in order to compete with other models. So when analysing model performance, the  specific type of data is not really relevant, as long as it matches these criteria.

\subsection{Deep neural networks forecast in electricity demand}

%More on the study, who, what, where etc.
%Some formulas or Graphs perhaps? More technical details, but not too much!

In a project report done at Stanford university in 2012[\citenum{StanfordNNReport}], a deep NN method was employed to forecast energy demand in the US. This kind of data, of which there are huge, easily accessible data sets, is particularly well suited for machine learning purposes. Furthermore, energy load data exhibit deep non-linear patterns, much like financial data does.

The study itself concluded in favour of deep learning techniques. With a Deep Recurrent NN, they obtained as high correlation rates as 99.6\%. In comparison, top industrial forecasters advertise mean percentage errors of between 0.84\% and 1.56\%. This puts the Deep NN on par with the conventional methods of the private sector.

The Deep recurrent NN responsible for the results, outperformed the other deep NN methods (ie. Deep Feedforward NN) used in the study by a wide margin. The authors point out that the other methods fail to incorporate some of the temporal structures found in time series data. These are data structures related to time intervals and points, essential to forecasting time series. This is of course relevant for other time series as well, such as financial market data.

In their conclusion, the authors contribute parts of their results to the nature of the data used, and the availability of it. Furthermore they state that by refining of their methods, and more specifically, by incorporating more advanced techniques in the process, the results could still improve. They specifically name the restricted Boltzmann machines as a more modern method they believe could out-perform their own top contender, a deep recurrent NN.


%Research Restricted Boltzmann machines
%Get Citation on underlying non-linear processes and correlations. Complicated processes.


\newpage

\section{Analysis}

\subsection{Analytical approach}

In this section the papers previously presented are dissected and their potentially relevant parts pointed out.
Each piece of research is analysed on its own, and the main points not put together until the conclusion section of the thesis.
 


\subsection{Comparing different Deep NNs}

\subsection{Deep NNs with contemporary linear methods}


\newpage

\section{Conclusion}

\subsection{Summary of analysis}

\subsection{Future development and possible outlooks}


%\clearpage                     % luku loppuu, loput kelluvat tänne, sivunv.

%\input{luku2}                  % tässä tyylissä ei sivunvaihtoja lukujen
%\input{luku3}                  %   välillä. Toiset ohjaajat haluavat 
%\input{luku4}                  %   sivunvaihdot.

\label{pages:text}
\clearpage                     % luku loppuu, loput kelluvat tänne, sivunvaihto
%\newpage                       % ellei ylempi tehoa, pakota lähdeluettelo 
                               % alkamaan uudelta sivulta

% -------------- Lähdeluettelo / reference list -----------------------
%
% Lähdeluettelo alkaa aina omalta sivultaan; pakota lähteet alkamaan
% joko \clearpage tai \newpage
%
%
% Muista, että saat kirjallisuusluettelon vasta
%  kun olet kääntänyt ja kaulinnut "latex, bibtex, latex, latex"
%  (ellet käytä Makefilea ja "make")

% Viitetyylitiedosto aaltosci_t.bst; muokattu HY:n tktl-tyylistä.
\bibliographystyle{aaltosci_t}
% Katso myös tämän tiedoston yläosan "preamble" ja siellä \bibpunct.

% Muutetaan otsikko "Kirjallisuutta" -> "Lähteet"
\renewcommand{\refname}{Sources}  % article-tyyppisen
%\renewcommand{\bibname}{Lähteet}  % jos olisi book, report-tyyppinen

% Lisätään sisällysluetteloon
\addcontentsline{toc}{section}{\refname}  % article
%\addcontentsline{toc}{chapter}{\bibname}  % book, report

% Määritä kaikki bib-tiedostot
\bibliography{lahteet}
%\bibliography{thesis_sources}

\label{pages:refs}
\clearpage         % erotetaan mahd. liitteet alkamaan uudelta sivulta

% -------------- Liitteet / Appendices --------------------------------
%
% Liitteitä ei yleensä tarvita. Kommentoi tällöin seuraavat
% rivit.

% Tiivistelmässä joskus matemaattisen kaavan tarkempi johtaminen, 
% haastattelurunko, kyselypohja, ylimääräisiä kuvia, lyhyitä 
% ohjelmakoodeja tai datatiedostoja.

%\appendix
%\input{luku_liitteet}

%\label{pages:appendices}

% ---------------------------------------------------------------------

\end{document}

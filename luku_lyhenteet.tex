% -------------- Symbolit ja lyhenteet --------------
%
% Suomen kielen lehtorin suositus: vasta kun noin 10-20 symbolia
% tai lyhennettä, niin käytä vasta sitten.
%
% Tämä voi puuttuakin. Toisaalta jos käytät paljon akronyymejä,
% niin ne kannattaa esitellä ensimmäisen kerran niitä käytettäessä.
% Muissa tapauksissa lukija voi helposti tarkistaa sen tästä
% luettelosta. Esim. "Automaattinen tietojenkäsittely (ATK) mahdollistaa..."
% "... ATK on ..."

\addcontentsline{toc}{section}{Käytetyt symbolit ja lyhenteet}

\section*{Used symbols and abbreviations}
%?? Käytetyt lyhenteet ja termit ??
%?? Käytetyt lyhenteet / termit / symbolit ??
%\section*{Abbreviations and Acronyms}

\begin{center}
\begin{tabular}{p{0.2\textwidth}p{0.65\textwidth}}

NN & Neural Network \\
RBM & Restricted Boltzmann Machine \\ 

\end{tabular}
\end{center}

\vspace{10mm}

%Jos tarvitset useampisivuista taulukkoa, kannattanee käyttää 
%esim. \verb!supertabular*!-ympäristöä, josta on kommentoitu esimerkki
%toisaalla tekstiä.


